%% This is file `elsarticle-template-1-num.tex',
%%
%% Copyright 2009 Elsevier Ltd
%%
%% This file is part of the 'Elsarticle Bundle'.
%% ---------------------------------------------
%%
%% It may be distributed under the conditions of the LaTeX Project Public
%% License, either version 1.2 of this license or (at your option) any
%% later version.  The latest version of this license is in
%%    http://www.latex-project.org/lppl.txt
%% and version 1.2 or later is part of all distributions of LaTeX
%% version 1999/12/01 or later.
%%
%% The list of all files belonging to the 'Elsarticle Bundle' is
%% given in the file `manifest.txt'.
%%
%% Template article for Elsevier's document class `elsarticle'
%% with numbered style bibliographic references
%%
%% $Id: elsarticle-template-1-num.tex 149 2009-10-08 05:01:15Z rishi $
%% $URL: http://lenova.river-valley.com/svn/elsbst/trunk/elsarticle-template-1-num.tex $
%%
\documentclass[preprint,12pt]{elsarticle}

%% Use the option review to obtain double line spacing
%% \documentclass[preprint,review,12pt]{elsarticle}

%% Use the options 1p,twocolumn; 3p; 3p,twocolumn; 5p; or 5p,twocolumn
%% for a journal layout:
%% \documentclass[final,1p,times]{elsarticle}
%% \documentclass[final,1p,times,twocolumn]{elsarticle}
%% \documentclass[final,3p,times]{elsarticle}
%% \documentclass[final,3p,times,twocolumn]{elsarticle}
%% \documentclass[final,5p,times]{elsarticle}
%% \documentclass[final,5p,times,twocolumn]{elsarticle}

%% if you use PostScript figures in your article
%% use the graphics package for simple commands
%% \usepackage{graphics}
%% or use the graphicx package for more complicated commands
%% \usepackage{graphicx}
%% or use the epsfig package if you prefer to use the old commands
%% \usepackage{epsfig}

%% The amssymb package provides various useful mathematical symbols
\usepackage{amssymb}
%% The amsthm package provides extended theorem environments
%% \usepackage{amsthm}
\usepackage{hyperref} %\url can be used, links are highlited


\usepackage[printonlyused]{acronym}

\usepackage{color}
\usepackage{xcolor}

\usepackage{listings} \lstset{numbers=left, numberstyle=\tiny,
  basicstyle=\footnotesize, stepnumber=2, numbersep=5pt,
  showspaces=false, showstringspaces=false, frame=single,
  breaklines=true,
  backgroundcolor=\color{white},morekeywords={id,gender, name,
    first_name, username, gender, locale},
  numbersep=5pt} \lstset{language=Perl} 




\newcounter{countNotes}
\newcommand{\annKB}[1]{\color{blue}*\textsuperscript{\thecountNotes}\marginpar{\color{blue} \tiny *\textsuperscript{\thecountNotes}\textbf{Note (KB): \\}
    #1  }\addtocounter{countNotes}{1}\color{black}}
%\renewcommand{\annKB}[1]{} %disable annotation 

% \renewcommand{sbw}{} %disable comments


%% The lineno packages adds line numbers. Start line numbering with
%% \begin{linenumbers}, end it with \end{linenumbers}. Or switch it on
%% for the whole article with \linenumbers after \end{frontmatter}.
%% \usepackage{lineno}

%% natbib.sty is loaded by default. However, natbib options can be
%% provided with \biboptions{...} command. Following options are
%% valid:

%%   round  -  round parentheses are used (default)
%%   square -  square brackets are used   [option]
%%   curly  -  curly braces are used      {option}
%%   angle  -  angle brackets are used    <option>
%%   semicolon  -  multiple citations separated by semi-colon
%%   colon  - same as semicolon, an earlier confusion
%%   comma  -  separated by comma
%%   numbers-  selects numerical citations
%%   super  -  numerical citations as superscripts
%%   sort   -  sorts multiple citations according to order in ref. list
%%   sort&compress   -  like sort, but also compresses numerical citations
%%   compress - compresses without sorting
%%
%% \biboptions{comma,round}

% \biboptions{}


\journal{has to be discussed}

\begin{document}

\begin{frontmatter}

%% Title, authors and addresses

%% use the tnoteref command within \title for footnotes;
%% use the tnotetext command for the associated footnote;
%% use the fnref command within \author or \address for footnotes;
%% use the fntext command for the associated footnote;
%% use the corref command within \author for corresponding author footnotes;
%% use the cortext command for the associated footnote;
%% use the ead command for the email address,
%% and the form \ead[url] for the home page:
%%
%% \title{Title\tnoteref{label1}}
%% \tnotetext[label1]{}
%% \author{Name\corref{cor1}\fnref{label2}}
%% \ead{email address}
%% \ead[url]{home page}
%% \fntext[label2]{}
%% \cortext[cor1]{}
%% \address{Address\fnref{label3}}
%% \fntext[label3]{}

\title{Distribution of a Facebook application}

%% use optional labels to link authors explicitly to addresses:
%% \author[label1,label2]{<author name>}
%% \address[label1]{<address>}
%% \address[label2]{<address>}


\author[focal]{Kathrin Becker\corref{cor1}\fnref{fn1}} 
\ead{mail@kathrin-becker.eu}
\author[focal]{Mayutan Arumaithurai\fnref{fn1}} %
\ead{mayutan.arumaithurai@gmail.com}
\author[focal]{Xiaoming Fu\corref{cor2}\fnref{fn1}} %
\ead{fu@cs.uni-goettingen.de}

\address[focal]{Institute of Computer Science, Computer Networks (NET)
  Research Group, University of G\"ottingen}





\cortext[cor1]{Corresponding author}
\cortext[cor2]{Principal corresponding author}
\fntext[fn1]{This document is a collaborative effort.}
%\fntext[fn2]{Another author footnote, but a little more longer.}
%\fntext[fn3]{Yet another author footnote. Indeed, you can have
%xany number of author footnotes.}
\tnotetext[tn1]{This is a specimen title.}



\begin{abstract}
%% Text of abstract, 150 - 350 words
% The outcome was students' perceptions of preparedness.
This paper investigates, how a small Facebook application 
distributes among friends. A graph is created which represents users, who
installed the application, as nodes and friendship between them as
edges. In addition, metrics are calculated to determine, which
parameters  make users likely to invite their friends to
participate. 

The application itself is a quiz which asks people 5
questions, randomly selected from a pool of 20 questions, 
about dentistry health facts. After the quiz, participants
get to know if they had a higher score than their friends and who of their
friends answered the questions they got correctly.

Results show users, who yield a high score in the quiz,
more likely to invent other users than users, who got a worse
score. In addition, the number of friends did not mark users to be
relevant for spreading the app. Indeed, people who had less friends
were much more likely to spread the application.
\end{abstract}

\begin{keyword}
%% keywords here, in the form: keyword \sep keyword

%% MSC codes here, in the form: \MSC code \sep code
%% or \MSC[2008] code \sep code (2000 is the default)
Facebook, Social Networking, data mining, social graph
\end{keyword}

\end{frontmatter}

%%

%% The Appendices part is started with the command \appendix;
%% appendix sections are then done as normal sections
%% \appendix
%% main text
%% Start line numbering here if you want
%%
%o \linenumbers

\section{Introduction}
\label{sec:introduction}
Founded in 2004, Facebook\footnote{www.facebook.com} has
become a social network currently used by more than 600 million people
all over the world\cite{facebook500}. Since 2007, Facebook provides an API allowing
third party developers to embed their own applications into
Facebook\footnote{http://developers.facebook.com/blog/archive}.
Subsequent to the F8 conference in April, 2010, Facebook simplified
access to its social graph via Open Graph Protocol \cite{facebookStats} and allowed
for integration of external websites. Until October, 2009,
over 350 000 apps had been seen on Facebook \cite{facebookBlog}.
Due to current statistics, people install about 20 million apps per day. In addition, ten thousand new
websites integrate with Facebook every day. \cite{facebookStats}

Due to integration in Facebook's social graph, third party
applications spare user authentication procedures and can access
private user information as well. These information can include basic
information and friend lists, but also interests and further data
specified by an access token  \footnote{http://developers.facebook.com/docs/authentication/}.
As the data provided by Facebook, and collected by analyzing user
behaviour on an application, represent a lot of social information, it
seems obvious, that these data also allow for studying social groups.

This paper investigates the distribution process of a quiz on Facebook
by analysing which parameters make users likely to distribute the application
and which parameters do not. 

In the first section, background information
about how applications can be integrated with Facebook are
given. Then, the methodology is described in section \ref{sec:method},
including the quiz, as well as the 
proceedings used for data collection and evaluation. In section
\ref{sec:results}, the results are detailed, and in section
\ref{sec:discussion} and \ref{sec:outlook}, they are discussed and an
outlook is given for further research. 


\section{Background}
\label{sec:background}
This section describes the social graph concept, how applications can be integrated with
Facebook, how authentication and authorisation is done, and in which
format data from the social graph can be obtained.

\subsection{Social Graph}
\label{sec:social-graph}
The \textit{social graph} is at Facebook's core. It represents
connections between people and things they care about
\cite{facebookDev}. In April, 2010, Facebook included access to the
social graph via the Open Graph Protocol, which advanced integration of
external websites into the social graph \cite{facebookDev2}. These websites can make use of
social plugins such as \textit{Like Buttons} or \textit{Comment Boxes} \cite{facebookDev2}
and for this reason, the social graph contains not only connections between
people on Facebook (friend relations), but also between 
entities such as applications, movie entries on external websites,
news articles, and much more. As this allows
people to identify friends on external websites and receive
information about their friends' attitude towards anything their
friends have visited and commented before, one can say
that Facebook's social graph makes the Internet social.  

The Graph API is one of Facebook's core concepts and presents a consistent
view of the social graph. Every object has an unique ID and its
properties can be accessed by requesting the appropriate URL:
\url{https://graph.facebook.com/ID}. Listing \ref{list:kbecker} shows
the author's public user node on Facebook's social graph. which is
returned as a \ac{json}-object\footnote{http://www.json.org}.

\begin{lstlisting}[caption=JSON object from
  \url{https://graph.facebook.com/681178092},
  label=list:kbecker]{}
{
   "id": "681178092",
   "name": "Kathrin Becker",
   "first_name": "Kathrin",
   "last_name": "Becker",
   "username": "kathrin.becker1",
   "gender": "female",
   "locale": "de_DE"
}
\end{lstlisting}

In order to get access not only to public, but also to private
information, an access token has to be included in the Graph API
request. To get this token, the accessing entity must be
authorized and authenticated to do so. How this can be achieved is
discussed in section \ref{sec:auth-auth}. If one tries to access
non-public information such as the author's list of friends, an error
in terms of a missing access token is returned. An example is given in
listing \ref{list:kbeckerfriends}.


\begin{lstlisting}[caption=JSON object from
  \url{https://graph.facebook.com/681178092/friends},
  label=list:kbeckerfriends]{}
{
   "error": {
      "type": "OAuthException",
      "message": "An access token is required to request this resource."
   }
}
\end{lstlisting}


Besides the Graph API, Facebook's social graph can also be accessed
with the REST-API, which is in the processes of deprecating
\cite{facebookDev3}. Both, the Graph API and the REST-API support
O-Auth 2.0 for authentication and authorization. O-Auth 2.0 is
included in Facebook's \ac{sdks} (see section
\ref{sec:integr-appl-with} and \ref{sec:auth-auth}).




\subsection{Integration of Applications with Facebook}
\label{sec:integr-appl-with}
Applications on Facebook are hosted on external severs, and are loaded
into a \textit{Canvas Page} by using iFrames. After registration
(\url{http://www.facebook.com/developers/createapp.php}), applications
can be found under the URL \url{http://apps.facebook.com/your_app }.
As soon as the application exists for some month and has
some users registered, it can be added to Facebook's application
index. 

Facebook provides several features to support the development
process. \ac{sdks} exist for popular programming languages and provide
sets of pre-defined functions for communication with Facebook. For
instance, they include the O-Auth 2.0 dialog which is described in
section \ref{sec:auth-auth}. In
addition, Facebook grants observation of its own development process by
making its raodmap public, and by showing its buglist. In addition, a
testserver exists to help developers to analyze if their application
is going to run with planned features from Facebook. 
Facebook provides an \textit{insights}-feature which makes
statistics about the applications available. \cite{oreilly}.



\subsection{Authentication and Authorization}
\label{sec:auth-auth}
In order to authorize and authenticate, Facebook recommends usage of
O-Auth 2.0 protocol by using the OAuth Dialog included in its
\ac{sdks}. As the application analyzed in this paper and described in section 
\ref{sec:method} implements the O-Auth Dialog from the PHP-\ac{sdk},
this variant will be described in detail. It shall just be mentioned,
that Facebook also allows for manual execution of the steps to receive
access tokens.\cite{facebookDevAuth}  

The OAuth Dialog performs three steps, i.e. user authentication, 
and authorization as well as authentication of the
application. If these three steps are completed, the application is
issued an \textit{user access token}. If more than basic information
is needed from Facebook, scope parameters can be set.\cite{facebookDevAuth}

Figure \ref{fig:authorization} details how authorization and
authentication take place with OAuthDialog. The \textit{app server}
illustrates the server on which the app is actually
running. \textit{Facebook} represents all the Facebook servers, on
which perform authentication and authorization. The \textit{OAuth
  Dialog} runs on the browser of the client who is going to use the
application.


\begin{figure}
\label{fig:authorization}
  \includegraphics[width=4in]{../img/fbinteraction3.png}
  \caption{Facebook establishes trust relation by not sending login
    data to app server, but using access tokens instead for
    data exchange with app server.}
\end{figure}

When clients access the canvas page of the application, they are
redirected to the OAuth Dialog (running on a Facebook server) and
provided with the application's ID. Then, user authentication and
app authorization is performed, by involving the client and Facebook
only. User authentication is done by logging in on Facebook (if necessary) and
authorization of the application is done by asking the user if he/she
wants to grant the application access to some data specified by the
application with the scope parameter. One should note, that no
login data are sent to the application, thus, this is how Facebook
establishes a trust relation. 
If the user authorizes the application, an \textit{authorization code} is returned and
the user is redirected to the application, to which the
\textit{authorization code} is forwarded. When registering the
application, an \textit{application secret} code has been
created. In order to authenticate the application, the
\textit{authorization code} and the \textit{application secret} have
to be sent to Facebook's \textit{Graph API Token Endpoint}. Facebook
then returns an \textit{access token}, which can be included in the
URL by which the Graph API can be accessed. However, Facebook's
\ac{sdks} also provide functions to perform calls to the Graph API,
and in this case, the access token is included automatically.


\section{Methodology}
\label{sec:method}
In this section, the quiz is described as well as its
implementation. Then, it is explained how data are collected for
further analysis and how they can be evaluated.

\subsection{The Quiz}
\label{sec:quiz}
The quiz contains questions in German about dentistry health facts. Participants are asked 5
questions taken from a pool of 20 questions. Each question has 4
possible answers. Multiple selection is possible, but at least one
answer is correct. A question is answered correctly, if each correct
and no incorrect answer have been selected. Figure \ref{fig:quiz}
shows how the quiz looks like. 


\begin{figure}
  \includegraphics[height=3in]{../img/fbquiz.png}
 \caption{The quiz.}
\label{fig:quiz}
\end{figure}


The quiz is competitive. After the quiz is completed, user are told
how good they were in comparison to other participants and to their
friends. As shown in figure \ref{fig:solution}, for each question the
correct solution is given. A user can also invite friends to
participate by clicking on the appropriate image. This is demonstrated in
figure \ref{fig:invite}.


\begin{figure}
  \includegraphics[width=3.5in]{../img/fbsolution.png}
  \caption{Solution of a question.}
\label{fig:solution}
\end{figure}



\begin{figure}
  \includegraphics[width=3.5in]{../img/fbinvite.png}
  \caption{This screenshot shows how friends can be invited.}
\label{fig:invite}
\end{figure}

\subsection{Implementation}
\label{sec:implementation}
The quiz-application detailed in the last section makes use of
Facebook's php-\ac{sdk} and JavaScript-\ac{sdk}. It requires
permissions for \textit{user$\_$birthday} and
\textit{publish$\_$stream}.
Data used by the quiz are stored in a MySQL-database, and php scripts
handle the application flow. \textit{index.php} maintains calls of
scripts printing the quiz (\textit{quiz.php}, \textit{question.php}), and displaying
the correct solution as well as the opportunity to invite friends
after completion of the quiz
(\textit{quizResult.php}). \textit{library.php} contains all the
functions needed.


\subsubsection{Database schema}
\label{sec:database-schema}
The database queries are performed with PHP
pear\footnote{http://pear.php.net/}-package. The following tables exist:
\begin{itemize}
\item \textbf{User} - entries are created for each user who participates as
well as for each of a user's friends. A flag details if user
participated or not.
\item \textbf{Friends} - creates entries for each friendship found in
users' friend lists.
\item \textbf{Questions} - contains the questions of the quiz.
\item \textbf{Answers} - contains the possible answers and a flag to mark them
as correct or false.
\item \textbf{GivenAnswers} - stores each answer given by a user.
\item \textbf{UserScore} - stores the score user had in the quiz.
\item \textbf{CorrectAnsweredQuestions} - stores which questions were correct
answered by which user, and which were not.
\end{itemize}

\subsubsection{Application flow}
\label{sec:application-flow}
When a user enters, the \textit{OAuth Dialog} is started. Then, the user data
including gender, birthday, name, Facebook-ID are stored together with
a timestamp, and column \textit{has-used-app} is set to 1. In
addition, each user's friends are also stored and column
\textit{has-used-app} is set to 0. Then, friendships are stored in the
table \textit{friends}. Next, the quiz is created by randomly choosing
five questions from the questions table and requiring the answers. To
make the questions more attractive, the name of an image convenient to
the questions topic is also stored in the questions table, and this
image is placed on the left side of each question. Since multiple
selection is possible, before the answers, which are also queried from
the MySQL database, checkboxes are placed before each answer. 

As soon as a user completed the quiz, evaluation starts by checking
which questions were answered correctly and by computing the final
score. Due to multiple selection, user answers are stored in
an multidimensional array and compared with another multidimensional
array containing the correct solution and created from the database.
The given answers are then stored, and in addition, in another table
(\textit{correctAnsweredQuestions)}
it is only stored which questions were correct and incorrect answered.
Some user join the quiz multiple times, and for this reason, each
score they had is stored in a score table. The user table is also
changed by entering or updating the average score, computed from each all the scores stored
in the score table. Then, by using the Java Script \ac{sdk}, the user
get the option to post the results onto their Facebook wall.

In order to make the whole thing challenging, it
is then computed how high the user's score is compared to the friends
and to all the other users. 
In addition, the correct solutions are given, and if a user knew the
right answer, a green check is placed left to the question. Otherwise,
a red cross is displayed at the same place. At least, the table with
the friendships and table \textit{correctAnsweredQuestions} are used
to identify friends  which answered the question correctly. Images of
them are placed beyond each question's solution together with the
information that those friends knew the right solution. 
In the end, users get the chance to invite friends to the app to
challenge them. For this reason, each friend's picture is added and
user can click at them. On click, a pre-defined function from
Facebook's Java Script \ac{sdk} is called to perform the invitation. 


\subsection{Data evaluation}
\label{sec:data-evaluation}
Data evaluation is also performed by using the MySQL tables, and
PHP. In addition, the dot
language\footnote{http://www.graphviz.org/Documentation.php} is used
to create graphs. 

The evaluation scripts are maintained by \textit{main.php}. A user
class represents Facebook's social graph internally. Properties of
user objects equal the \textit{user}-table's properties, and a user's
friendships are represented by a list of references to corresponding
user objects. The whole set of users is maintained by
\textit{main.php}, where they are stored in an array. Class \textit{Graph}
allows for output of various graphs in dot-language created from the
internal social graph. 

\section{Results}
\label{sec:results}


% \begin{center}
% \begin{table}
% \label{tab:coruser}
% \begin{tabular}{l | c c c}
% a & a & a & a \\

% \hline
% b & b & b & b
% \end{tabular}
% \caption{dummy}
% \end{table}
% \end{center}





\section{Discussion}
\label{sec:discussion}




\section{Outlook}
\label{sec:outlook}


%\section{Conclusion}
%\label{sec:conclusion}




%%
%% This is file `acronym.sty',
%% generated with the docstrip utility.
%%
%% The original source files were:
%%
%% acronym.dtx  (with options: `acronym')
%%  Copyright 1995--2005  by Tobias Oetiker (oetiker@ee.ethz.ch)
%%                        and individual authors listed elsewhere.
%%  All rights reserved.
%% 
%%  This work may be distributed and/or modified under the conditions of
%%  the LaTeX Project Public License, either version 1.3 of this license
%%  or (at your option) any later version. The latest version of the
%%  license is in
%% 
%%      http://www.latex-project.org/lppl.txt
%% 
%%  and version 1.3 or later is part of all distributions of LaTeX
%%  version 2003/12/01 or later.
%% 
%%  This work has the LPPL maintenance status "maintained".
%%  The Current Maintainer of this work is Tobias Oetiker (oetiker@ee.ethz.ch).
%% 
%% \CharacterTable
%%  {Upper-case    \A\B\C\D\E\F\G\H\I\J\K\L\M\N\O\P\Q\R\S\T\U\V\W\X\Y\Z
%%   Lower-case    \a\b\c\d\e\f\g\h\i\j\k\l\m\n\o\p\q\r\s\t\u\v\w\x\y\z
%%   Digits        \0\1\2\3\4\5\6\7\8\9
%%   Exclamation   \!     Double quote  \"     Hash (number) \#
%%   Dollar        \$     Percent       \%     Ampersand     \&
%%   Acute accent  \'     Left paren    \(     Right paren   \)
%%   Asterisk      \*     Plus          \+     Comma         \,
%%   Minus         \-     Point         \.     Solidus       \/
%%   Colon         \:     Semicolon     \;     Less than     \<
%%   Equals        \=     Greater than  \>     Question mark \?
%%   Commercial at \@     Left bracket  \[     Backslash     \\
%%   Right bracket \]     Circumflex    \^     Underscore    \_
%%   Grave accent  \`     Left brace    \{     Vertical bar  \|
%%   Right brace   \}     Tilde         \~}
%%
\NeedsTeXFormat{LaTeX2e}[1999/12/01]
\ProvidesPackage{acronym}[2009/01/25
                          v1.34
                          Support for acronyms (Tobias Oetiker)]
\RequirePackage{suffix}
\newif\ifAC@footnote
\AC@footnotefalse
\DeclareOption{footnote}{\AC@footnotetrue}
\newif\ifAC@nohyperlinks
\AC@nohyperlinksfalse
\DeclareOption{nohyperlinks}{\AC@nohyperlinkstrue}
\newif\ifAC@printonlyused
\AC@printonlyusedfalse
\DeclareOption{printonlyused}{\AC@printonlyusedtrue}
\newif\ifAC@withpage
\AC@withpagefalse
\DeclareOption{withpage}{\AC@withpagetrue}
\newif\ifAC@smaller
\AC@smallerfalse
\DeclareOption{smaller}{\AC@smallertrue}
\newif\ifAC@dua
\AC@duafalse
\DeclareOption{dua}{\AC@duatrue}
\newif\ifAC@nolist
\AC@nolistfalse
\DeclareOption{nolist}{\AC@nolisttrue}
\ProcessOptions\relax
\ifAC@smaller
  \RequirePackage{relsize}
  \newcommand*{\acsfont}[1]{\textsmaller{#1}}
\else
  \newcommand*{\acsfont}[1]{#1}
\fi
\newcommand*{\acffont}[1]{#1}
\newcommand*{\acfsfont}[1]{#1}
\def\AC@hyperlink#1#2{#2}
\def\AC@hypertarget#1#2{#2}
\def\AC@phantomsection{}
\ifAC@nohyperlinks
\else
   \AtBeginDocument{%
      \@ifpackageloaded{hyperref}
         {\let\AC@hyperlink=\hyperlink
          \newcommand*\AC@raisedhypertarget[2]{%
             \Hy@raisedlink{\hypertarget{#1}{}}#2}%
          \let\AC@hypertarget=\AC@raisedhypertarget
          \def\AC@phantomsection{%
            \Hy@GlobalStepCount\Hy@linkcounter
            \edef\@currentHref{section*.\the \Hy@linkcounter}%
            \Hy@raisedlink{%
              \hyper@anchorstart{\@currentHref}\hyper@anchorend
            }%
          }%
         }{}}%
\fi
\AtBeginDocument{%
   \providecommand\texorpdfstring[2]{#1}%
   \providecommand\pdfstringdefDisableCommands[1]{}%
   \pdfstringdefDisableCommands{%
     \csname AC@starredfalse\endcsname
     \csname AC@footnotefalse\endcsname
     \let\AC@hyperlink\@secondoftwo
     \let\acsfont\relax
     \let\acffont\relax
     \let\acfsfont\relax
     \let\acused\relax
     \let\null\relax
     \def\AChy@call#1#2{%
        \ifx*#1\@empty
          \expandafter #2%
        \else
          #2{#1}%
        \fi
      }%
      \def\acs#1{\AChy@call{#1}\AC@acs}%
      \def\acl#1{\AChy@call{#1}\@acl}%
      \def\acf#1{\AChy@call{#1}\AChy@acf}%
      \def\ac#1{\AChy@call{#1}\@ac}%
      \def\acsp#1{\AChy@call{#1}\@acsp}%
      \def\aclp#1{\AChy@call{#1}\@aclp}%
      \def\acfp#1{\AChy@call{#1}\AChy@acfp}%
      \def\acp#1{\AChy@call{#1}\@acp}%
      \def\acfi#1{\AChy@call{#1}\AChy@acf}%
      \let\acsu\acs
      \let\aclu\acl
      \def\AChy@acf#1{\AC@acl{#1} (\AC@acs{#1})}%
      \def\AChy@acfp#1{\AC@acl{#1}s (\AC@acs{#1}s)}%
   }%
}
\newtoks\AC@clearlist
\newcommand*\AC@addtoAC@clearlist[1]{%
  \global\AC@clearlist\expandafter{\the\AC@clearlist\AC@reset{#1}}%
}
\newcommand*\acresetall{\the\AC@clearlist\AC@clearlist={}}
\def\AC@reset#1{%
  \global\expandafter\let\csname ac@#1\endcsname\relax
}
\newcommand*\AC@used{@<>@<>@}
\newcommand{\AC@populated}{}
\newcommand*{\AC@logged}[1]{%
   \acronymused{#1}% mark it as used in the current run too
   \@bsphack
   \protected@write\@auxout{}{\string\acronymused{#1}}%
   \@esphack}
\AtBeginDocument{%
   \pdfstringdefDisableCommands{%
      \let\AC@logged\@gobble
   }%
}
\newcommand*{\acronymused}[1]{%
   \expandafter\ifx\csname acused@#1\endcsname\AC@used
      \relax
   \else
       \global\expandafter\let\csname acused@#1\endcsname\AC@used
       \global\let\AC@populated\AC@used
   \fi}
\newcommand*\newacro[1]{%
  \@ifnextchar[{\AC@newacro{#1}}{\AC@newacro{#1}[\AC@temp]}}
\newcommand\AC@newacro{}
\def\AC@newacro#1[#2]#3{%
   \def\AC@temp{#1}%
   \expandafter\gdef\csname fn@#1\endcsname{{#2}{#3}}%
   }
\newcommand*\acrodef[1]{%
  \@ifnextchar[{\AC@acrodef{#1}}{\AC@acrodef{#1}[\AC@temp]}}
\newcommand\AC@acrodef{}
\def\AC@acrodef#1[#2]#3{%
   \def\AC@temp{#1}%
   \@bsphack
   \protected@write\@auxout{}{\string\newacro{#1}[#2]{#3}}%
   \@esphack}
\def\bflabel#1{{\textbf{\textsf{#1}}\hfill}}
\newenvironment{AC@deflist}[1]%
        {\ifAC@nolist%
         \else%
            \raggedright\begin{list}{}%
                {\settowidth{\labelwidth}{\textbf{\textsf{#1}}}%
                \setlength{\leftmargin}{\labelwidth}%
                \addtolength{\leftmargin}{\labelsep}%
                \renewcommand{\makelabel}{\bflabel}}%
          \fi}%
        {\ifAC@nolist%
         \else%
            \end{list}%
         \fi}%
\newcommand{\acroextra}[1]{}
\newenvironment{acronym}[1][1]{%
   \providecommand*{\acro}{\AC@acro}%
   \long\def\acroextra##1{##1}%
   \def\@tempa{1}\def\@tempb{#1}%
   \ifx\@tempa\@tempb%
      \global\expandafter\let\csname ac@des@mark\endcsname\AC@used%
      \ifAC@nolist%
      \else%
         \begin{description}%
      \fi%
   \else%
      \begin{AC@deflist}{#1}%
   \fi%
  }%
  {%
   \ifx\AC@populated\AC@used\else%
      \ifAC@nolist%
      \else%
          \item[]\relax%
      \fi%
   \fi%
   \expandafter\ifx\csname ac@des@mark\endcsname\AC@used%
      \ifAC@nolist%
      \else%
        \end{description}%
      \fi%
   \else%
      \end{AC@deflist}%
   \fi}%
\newcommand*\AC@acro[1]{%
  \@ifnextchar[{\AC@@acro{#1}}{\AC@@acro{#1}[\AC@temp]}}
\newcommand\AC@@acro{}
\def\AC@@acro#1[#2]#3{%
  \def\AC@temp{#1}%
  \ifAC@nolist%
  \else%
  \ifAC@printonlyused%
    \expandafter\ifx\csname acused@#1\endcsname\AC@used%
       \item[\protect\AC@hypertarget{#1}{\acsfont{#2}}] #3%
          \ifAC@withpage%
            \expandafter\ifx\csname r@acro:#1\endcsname\relax%
               \PackageInfo{acronym}{%
                 Acronym #1 used in text but not spelled out in
                 full in text}%
            \else%
               \dotfill\pageref{acro:#1}%
            \fi\\%
          \fi%
    \fi%
 \else%
    \item[\protect\AC@hypertarget{#1}{\acsfont{#2}}] #3%
 \fi%
 \fi%
 \begingroup
    \def\acroextra##1{}%
    \@bsphack
    \protected@write\@auxout{}%
       {\string\newacro{#1}[\string\AC@hyperlink{#1}{#2}]{#3}}%
    \@esphack
  \endgroup}
\newif\ifAC@starred
\newcommand*\AC@get[3]{%
    \ifx#1\relax
       \PackageWarning{acronym}{Acronym `#3' is not defined}%
       \textbf{#3!}%
    \else
       \expandafter#2#1\null
    \fi}
\newcommand*\AC@acs[1]{%
   \expandafter\AC@get\csname fn@#1\endcsname\@firstoftwo{#1}}
\newcommand*\AC@acl[1]{%
   \expandafter\AC@get\csname fn@#1\endcsname\@secondoftwo{#1}}
\newcommand*{\acs}{\AC@starredfalse\protect\acsa}%
\WithSuffix\newcommand\acs*{\AC@starredtrue\protect\acsa}%
\newcommand*{\acsa}[1]{%
   \texorpdfstring{\protect\@acs{#1}}{#1}}
\newcommand*{\@acs}[1]{%
   \acsfont{\AC@acs{#1}}%
%% having a footnote on acs sort of defetes the purpose
%%   \ifAC@footnote
%%      \footnote{\AC@acl{#1}{}}%
%%   \fi
   \ifAC@starred\else\AC@logged{#1}\fi}
\newcommand*{\acl}{\AC@starredfalse\protect\@acl}%
\WithSuffix\newcommand\acl*{\AC@starredtrue\protect\@acl}%
\newcommand*{\@acl}[1]{%
   \AC@acl{#1}%
   \ifAC@starred\else\AC@logged{#1}\fi}
\newcommand*\@verridelabel[1]{%
  \@bsphack
  \protected@write\@auxout{}{\string\undonewlabel{#1}}%
  \label{#1}%
  \@overriddenmessage rs{#1}%
  \@esphack
}%
\newcommand*\undonewlabel{\@und@newl@bel rs}%
\newcommand*\@und@newl@bel[3]{%
  \@ifundefined{#1@#3}%
  {%
    \global\expandafter\let\csname#2@#3\endcsname\@nnil
  }%
  {%
    \global\expandafter\let\csname#1@#3\endcsname\relax
  }%
}%
\newcommand*\@overriddenmessage[3]{%
  \expandafter\ifx\csname#2@#3\endcsname\@nnil
    \expandafter\@firstoftwo
  \else
    \@ifundefined{#1@#3}%
    {%
      \@ifundefined{#2@#3}%
      {\expandafter\@firstoftwo}%
      {\expandafter\@secondoftwo}%
    }%
    {\expandafter\@secondoftwo}%
  \fi
  {%
    \PackageInfo{acronym}{Label `#3' newly defined as it
    shall be overridden^^Jalthough it is yet undefined}%
    \global\expandafter\let\csname#2@#3\endcsname\empty
  }%
  {%
    \PackageInfo{acronym}{Label `#3' overridden}%
    \@ifundefined{#2@#3}{%
      \global\expandafter\let\csname#2@#3\endcsname\empty}{}%
    \expandafter\g@addto@macro\csname#2@#3\endcsname{i}%
  }%
}%
\newcommand*\ac@testdef[3]{%
  \@ifundefined{s@#2}\@secondoftwo\@firstofone
  {%
    \expandafter\ifx\csname s@#2\endcsname\empty
      \expandafter\@firstofone
    \else
      \expandafter\xdef\csname s@#2\endcsname{%
        \expandafter\expandafter
        \expandafter\@gobble
        \csname s@#2\endcsname
      }%
      \expandafter\@gobble
    \fi
  }%
  {%
    \@testdef{#1}{#2}{#3}%
  }%
}%
\protected@write\@auxout{}{%
  \string\reset@newl@bel
}%
\newcommand*\reset@newl@bel{%
  \ifx\@newl@bel\@testdef
    \let\@newl@bel\ac@testdef
    \let\undonewlabel\@gobble
  \fi
}%
\newcommand*\AC@placelabel[1]{%
  \expandafter\ifx\csname ac@#1\endcsname\AC@used
  \else
    {\AC@phantomsection\@verridelabel{acro:#1}}%
    \global\expandafter\let\csname ac@#1\endcsname\AC@used
    \AC@addtoAC@clearlist{#1}%
  \fi
}%
\newcommand*{\acf}{\AC@starredfalse\protect\acfa}%
\WithSuffix\newcommand\acf*{\AC@starredtrue\protect\acfa}%
\newcommand*{\acfa}[1]{%
   \texorpdfstring{\protect\@acf{#1}}{\AC@acl{#1} (#1)}}
\newcommand*{\@acf}[1]{%
    \ifAC@footnote
       \acsfont{\AC@acs{#1}}%
       \footnote{\AC@placelabel{#1}\AC@acl{#1}{}}%
    \else
       \acffont{%
          \AC@placelabel{#1}\AC@acl{#1}%
          \nolinebreak[3] %
          \acfsfont{(\acsfont{\AC@acs{#1}})}%
        }%
     \fi
     \ifAC@starred\else\AC@logged{#1}\fi}
\newcommand*{\ac}{\AC@starredfalse\protect\@ac}%
\WithSuffix\newcommand\ac*{\AC@starredtrue\protect\@ac}%
\newcommand{\@ac}[1]{%
  \ifAC@dua
     \ifAC@starred\acl*{#1}\else\acl{#1}\fi%
  \else
     \expandafter\ifx\csname ac@#1\endcsname\AC@used%
     \ifAC@starred\acs*{#1}\else\acs{#1}\fi%
   \else
     \ifAC@starred\acf*{#1}\else\acf{#1}\fi%
   \fi
  \fi}
\newcommand*{\acsp}{\AC@starredfalse\protect\acspa}%
\WithSuffix\newcommand\acsp*{\AC@starredtrue\protect\acspa}%
\newcommand*{\acspa}[1]{%
   \texorpdfstring{\protect\@acsp{#1}}{#1s}}
\newcommand*{\@acsp}[1]{%
   \acsfont{\AC@acs{#1}s}%
   \ifAC@starred\else\AC@logged{#1}\fi}
\newcommand*{\aclp}{\AC@starredfalse\protect\@aclp}%
\WithSuffix\newcommand\aclp*{\AC@starredtrue\protect\@aclp}%
\newcommand*{\@aclp}[1]{%
   \AC@acl{#1}s%
   \ifAC@starred\else\AC@logged{#1}\fi}
\newcommand*{\acfp}{\AC@starredfalse\protect\acfpa}%
\WithSuffix\newcommand\acfp*{\AC@starredtrue\protect\acfpa}%
\newcommand*{\acfpa}[1]{%
   \texorpdfstring{\protect\@acfp{#1}}{\AC@acl{#1}s (#1s)}}
\newcommand*{\@acfp}[1]{%
   \ifAC@footnote
      \acsfont{\AC@acs{#1}s}%
      \footnote{\AC@placelabel{#1}\AC@acl{#1}s{}}%
   \else
      \acffont{%
         \AC@placelabel{#1}\AC@acl{#1}s%
         \nolinebreak[3] %
         \acfsfont{(\acsfont{\AC@acs{#1}s})}%
         }%
   \fi
   \ifAC@starred\else\AC@logged{#1}\fi}
\newcommand*{\acp}{\AC@starredfalse\protect\@acp}%
\WithSuffix\newcommand\acp*{\AC@starredtrue\protect\@acp}%
\newcommand{\@acp}[1]{%
  \ifAC@dua
     \ifAC@starred\aclp*{#1}\else\aclp{#1}\fi%
  \else
   \expandafter\ifx\csname ac@#1\endcsname\AC@used
      \ifAC@starred\acsp*{#1}\else\acsp{#1}\fi%
   \else
      \ifAC@starred\acfp*{#1}\else\acfp{#1}\fi%
   \fi
  \fi}
\newcommand*{\acfi}{\AC@starredfalse\protect\acfia}%
\WithSuffix\newcommand\acfi*{\AC@starredtrue\protect\acfia}%
\newcommand{\acfia}[1]{%
  {\itshape \AC@acl{#1} \nolinebreak[3]} (\ifAC@starred\acs*{#1}\else\acs{#1}\fi)}
\newcommand{\acused}[1]{%
\global\expandafter\let\csname ac@#1\endcsname\AC@used%
\AC@addtoAC@clearlist{#1}}
\newcommand*{\acsu}{\AC@starredfalse\protect\acsua}%
\WithSuffix\newcommand\acsu*{\AC@starredtrue\protect\acsua}%
\newcommand{\acsua}[1]{%
   \ifAC@starred\acs*{#1}\else\acs{#1}\fi\acused{#1}}
\newcommand*{\aclu}{\AC@starredfalse\protect\aclua}%
\WithSuffix\newcommand\aclu*{\AC@starredtrue\protect\aclua}%
\newcommand{\aclua}[1]{%
   \ifAC@starred\acl*{#1}\else\acl{#1}\fi\acused{#1}}
\endinput
%%
%% End of file `acronym.sty'.
 % load acronyms
%% \section{}
%% \label{}

%% References
%%
%% Following citation commands can be used in the body text:
%% Usage of \cite is as follows:
%%   \cite{key}          ==>>  [#]
%%   \cite[chap. 2]{key} ==>>  [#, chap. 2]
%%   \citet{key}         ==>>  Author [#]

%% References with bibTeX database:

\bibliographystyle{elsarticle-harv}
\bibliography{literature.bib}


%% Authors are advised to submit their bibtex database files. They are
%% requested to list a bibtex style file in the manuscript if they do
%% not want to use model1-num-names.bst.

%% References without bibTeX database:

% \begin{thebibliography}{00}

%% \bibitem must have the following form:
%%   \bibitem{key}...
%%

% \bibitem{}

% \end{thebibliography}


\end{document}

%%
%% End of file `elsarticle-template-1-num.tex'.

%%% Local Variables: 
%%% mode: latex
%%% TeX-master: t
%%% End: 
