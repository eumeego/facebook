%% This is file `elsarticle-template-1-num.tex',
%%
%% Copyright 2009 Elsevier Ltd
%%
%% This file is part of the 'Elsarticle Bundle'.
%% ---------------------------------------------
%%
%% It may be distributed under the conditions of the LaTeX Project Public
%% License, either version 1.2 of this license or (at your option) any
%% later version.  The latest version of this license is in
%%    http://www.latex-project.org/lppl.txt
%% and version 1.2 or later is part of all distributions of LaTeX
%% version 1999/12/01 or later.
%%
%% The list of all files belonging to the 'Elsarticle Bundle' is
%% given in the file `manifest.txt'.
%%
%% Template article for Elsevier's document class `elsarticle'
%% with numbered style bibliographic references
%%
%% $Id: elsarticle-template-1-num.tex 149 2009-10-08 05:01:15Z rishi $
%% $URL: http://lenova.river-valley.com/svn/elsbst/trunk/elsarticle-template-1-num.tex $
%%
\documentclass[preprint,12pt]{elsarticle}

%% Use the option review to obtain double line spacing
%% \documentclass[preprint,review,12pt]{elsarticle}

%% Use the options 1p,twocolumn; 3p; 3p,twocolumn; 5p; or 5p,twocolumn
%% for a journal layout:
%% \documentclass[final,1p,times]{elsarticle}
%% \documentclass[final,1p,times,twocolumn]{elsarticle}
%% \documentclass[final,3p,times]{elsarticle}
%% \documentclass[final,3p,times,twocolumn]{elsarticle}
%% \documentclass[final,5p,times]{elsarticle}
%% \documentclass[final,5p,times,twocolumn]{elsarticle}

%% if you use PostScript figures in your article
%% use the graphics package for simple commands
%% \usepackage{graphics}
%% or use the graphicx package for more complicated commands
%% \usepackage{graphicx}
%% or use the epsfig package if you prefer to use the old commands
%% \usepackage{epsfig}

%% The amssymb package provides various useful mathematical symbols
\usepackage{amssymb}
%% The amsthm package provides extended theorem environments
%% \usepackage{amsthm}
\usepackage{hyperref} %\url can be used, links are highlited

%\usepackage[latin1]{inputenc} % Zeichencodierung
%\usepackage[T1]{fontenc} % Zeichencodierung

\usepackage[printonlyused]{acronym}

\usepackage{color}
\usepackage{xcolor}

\usepackage{listings} \lstset{numbers=left, numberstyle=\tiny,
  basicstyle=\footnotesize, stepnumber=2, numbersep=5pt,
  showspaces=false, showstringspaces=false, frame=single,
  breaklines=true,
  backgroundcolor=\color{white},morekeywords={id,gender, name,
    first_name, username, gender, locale},
  numbersep=5pt} \lstset{language=Perl} 




\newcounter{countNotes}
\newcommand{\annKB}[1]{\color{blue}*\textsuperscript{\thecountNotes}\marginpar{\color{blue} \tiny *\textsuperscript{\thecountNotes}\textbf{Note (KB): \\}
    #1  }\addtocounter{countNotes}{1}\color{black}}
%\renewcommand{\annKB}[1]{} %disable annotation 

% \renewcommand{sbw}{} %disable comments


%% The lineno packages adds line numbers. Start line numbering with
%% \begin{linenumbers}, end it with \end{linenumbers}. Or switch it on
%% for the whole article with \linenumbers after \end{frontmatter}.
%% \usepackage{lineno}

%% natbib.sty is loaded by default. However, natbib options can be
%% provided with \biboptions{...} command. Following options are
%% valid:

%%   round  -  round parentheses are used (default)
%%   square -  square brackets are used   [option]
%%   curly  -  curly braces are used      {option}
%%   angle  -  angle brackets are used    <option>
%%   semicolon  -  multiple citations separated by semi-colon
%%   colon  - same as semicolon, an earlier confusion
%%   comma  -  separated by comma
%%   numbers-  selects numerical citations
%%   super  -  numerical citations as superscripts
%%   sort   -  sorts multiple citations according to order in ref. list
%%   sort&compress   -  like sort, but also compresses numerical citations
%%   compress - compresses without sorting
%%
%% \biboptions{comma,round}

% \biboptions{}


\journal{has to be discussed}

\begin{document}

\begin{frontmatter}

%% Title, authors and addresses

%% use the tnoteref command within \title for footnotes;
%% use the tnotetext command for the associated footnote;
%% use the fnref command within \author or \address for footnotes;
%% use the fntext command for the associated footnote;
%% use the corref command within \author for corresponding author footnotes;
%% use the cortext command for the associated footnote;
%% use the ead command for the email address,
%% and the form \ead[url] for the home page:
%%
%% \title{Title\tnoteref{label1}}
%% \tnotetext[label1]{}
%% \author{Name\corref{cor1}\fnref{label2}}
%% \ead{email address}
%% \ead[url]{home page}
%% \fntext[label2]{}
%% \cortext[cor1]{}
%% \address{Address\fnref{label3}}
%% \fntext[label3]{}

\title{Spreading of a Facebook Application}

%% use optional labels to link authors explicitly to addresses:
%% \author[label1,label2]{<author name>}
%% \address[label1]{<address>}
%% \address[label2]{<address>}


\author[focal]{Kathrin Becker\corref{cor1}\fnref{fn1}} 
\ead{mail@kathrin-becker.eu}
\author[focal]{Mayutan Arumaithurai\fnref{fn1}} %
\ead{mayutan.arumaithurai@gmail.com}
\author[focal]{Xiaoming Fu\corref{cor2}\fnref{fn1}} %
\ead{fu@cs.uni-goettingen.de}

\address[focal]{Institute of Computer Science, Computer Networks (NET)
  Research Group, University of G\"ottingen}





%\cortext[cor1]{Corresponding author}
%\cortext[cor2]{Principal corresponding author}
%\fntext[fn1]{This document is a collaborative effort.}
%\fntext[fn2]{Another author footnote, but a little more longer.}
%\fntext[fn3]{Yet another author footnote. Indeed, you can have
%xany number of author footnotes.}
%\tnotetext[tn1]{This is a specimen title.}



\begin{abstract}
%% Text of abstract, 150 - 350 words
% The outcome was students' perceptions of preparedness.
Applications on Facebook are popular and attract many
people. Nevertheless, prior to installation people must acknowledge that an application
exists, and they must demand the application, too.
There may exist several reasons for people to install
applications, such as intensive usage within their social network or
personal interest in a special topic. Persuasion can
happen virally by Facebook's news stream or by marketing through
friends, but also through directed search on 
Facebook. Differences in the spreading process are supposed to exist
for the different kinds of applications. 

This paper investigates how a quiz application spreads on
Facebook. A methodology is developed to conclude from a graph of
participants which people were very likely responsible for the application's spreading.

The quiz is about dentistry health facts and asks its participants five randomly
selected questions. In addition, it is endowed with a social component: After
completion of the quiz participants do not get the solution solely. Instead,
beyond the solution images of friends are displayed who knew the right answer of the
questions the user got. In addition, users are told how good their 
score was in comparison to their friends and to the whole set of users. 

Analysis of the applications way of spreading is supported by a tool maintaining an internal representation
of an extract of Facebook's social graph. The tool's internal graph
represents the participants as nodes, their friendships as edges, and their basic information
retrieved from Facebook as well as quiz results as
attributes of the nodes. The methodology to identify people who
contributed to the spreading process was implemented, too. The tool
allows for output of aspects of the internal graph 
in DOT-language, and export of metrics to CSV, too. 

Results show that participants who are supposed to contribute to the
application's spreading often achieve a comparatively high score in the
quiz. Particular user's posts in the news stream seem to influence the
applications growth, while other users post do not show this
effect. New participants often had strong relations to the network of people who
already participated. Indeed, the number of friends does not seem to
have any effect on whether a user gathers new participants for the quiz.
\end{abstract}

\begin{keyword}
%% keywords here, in the form: keyword \sep keyword

%% MSC codes here, in the form: \MSC code \sep code
%% or \MSC[2008] code \sep code (2000 is the default)
Facebook, Spreading of Applications, Social Network, data mining, social graph
\end{keyword}

\end{frontmatter}

%%

%% The Appendices part is started with the command \appendix;
%% appendix sections are then done as normal sections
%% \appendix
%% main text
%% Start line numbering here if you want
%%
%o \linenumbers

\section{Introduction}
\label{sec:introduction}
Founded in 2004 Facebook\footnote{www.facebook.com on 14th of April 2011} has
become a social network currently used by more than 600 million people
all over the world\cite{facebook500}. Since 2007, Facebook provides an API, permitting
third party developers to embed their own applications into
Facebook.
Subsequent to the F8 conference in April, 2010, Facebook simplifies
access to its social graph by allowing
external websites for integration in their social graph through Open
Graph Protocol.\cite{facebookStats} 

Due to integration in Facebook's social graph, third party
applications not only spare user from authentication procedures, but can also access
extracts of Facebook's social graph, too. This information details a
user's friend list, basic information, and 
further data. Access to this data has to be granted by access tokens. \cite{facebookDevAuth}
Thus, applications let Facebook's social graph enlarge and make people 
spend more time on Facebook. In addition, application developers
can spread their application virally in news
streams, and thereby they collect useful information about the social network of
their users.

Statistics show that the development of Facebook applications is becoming more
and more popular. Until October 2009, 
over three hundred fifty thousand applications have appeared on Facebook \cite{facebookBlog}.
According to current statistics about twenty million applications are
installed every day, and ten thousand new websites integrate with
Facebook every day. \cite{facebookStats} 

This paper investigates the way in which an application spreads on
Facebook. For this purpose, a quiz was built asking users five randomly
selected questions about dentistry health facts. User data are
collected in a MySQL database. The Evaluation of the data is
accomplished through the output of a tool which maintains an internal
representation of the extract of Facebook's 
social graph the application is allowed to access.

The tool allows for the output of graphs in \ac{dot} language and the output of
metrics as \ac{csv}. The graphs the tool puts out detail how the application
spreads, and allow for labelling of people (nodes) that connect the
network of existing user to new networks. The metrics detail the
number of friends users have, the average score,
and finally, the amount of users gained through the participant.

In the first section, basic information about Facebook's social graph,
and integration of applications is
given. In the next section, the methodology is described by detailing
the quiz and its implementation, as well as a tool created for data
evaluation. Section 
\ref{sec:results} presents the results, and in section
\ref{sec:discussion}, a discussion follows. Section
\ref{sec:conclusion} concludes the paper by summing up the results,
and giving an outlook on further research.


\section{Foundations}
\label{sec:background}
This section details concepts from Facebook, basics about the
\ac{dot} language, and finally, it defines the metrics the
data-evaluation tool (see section \ref{sec:method}) can put out. Facebook's
social graph concept is explained in 
section \ref{sec:social-graph}, section \ref{sec:integr-appl-with}
describes how applications can be integrated with Facebook, and
section \ref{sec:auth-auth} illustrates the way how applications can be
integrated with Facebook.

\subsection{Facebook's Social Graph}
\label{sec:social-graph}
Facebook's \textit{social graph} can be regarded as its core. It
represents connections between people and things people care about 
\cite{facebookDev}. Since April 2010, Facebook allows access to its
social graph via the Open Graph Protocol\footnote{http://ogp.me/}, thus facilitating
external websites and third party applications to integrate in the social graph by becoming
additional nodes\cite{facebookDev2}. For instance, these additional
nodes can represent entities such as movies, news articles, 
applications, and many more. People on Facebook can add edges
between their user node to these additional nodes by clicking on plugins
provided by Facebook, such as \textit{Like Buttons} or entering content in \textit{Comment
  Boxes}.  \cite{facebookDev2} As images of friends liking an
entity, and comments of friends can be shown on external websites via
iframes, aspects from a user's social network are brought to these
websites. Therefore, Facebook is said to socialize the Internet.

In general, additional nodes representing external websites are integrated
unidirectionally. Thus, Facebook retrieves the information about the
entities users were interested in, whereas the external websites do not
gain access to Facebook's social graph. Instead, the social
information is included via iframes.
In order to integrate nodes bi-directionally, the user must permit 
access to its data, and both,  user and entity, wanting to gain access
to the social graph, must be authenticated. This process is detailed in
section \ref{sec:auth-auth} and matters in the case of applications on
Facebook. Thus, applications are included bi-directional and can grab
information from Facebook's social graph.

The Graph API is one of Facebook's core concepts and presents a consistent
view of the social graph. Every object has an unique ID and if it is
public, its properties can be accessed by requesting the appropriate URL:
\url{https://graph.facebook.com/ID}. Listing \ref{list:kbecker} shows
the author's public user node on Facebook's social graph. which is
returned as a \ac{json}-object\footnote{http://www.json.org}.

\begin{lstlisting}[caption=JSON object from
  \url{https://graph.facebook.com/681178092},
  label=list:kbecker]{}
{
   "id": "681178092",
   "name": "Kathrin Becker",
   "first_name": "Kathrin",
   "last_name": "Becker",
   "username": "kathrin.becker1",
   "gender": "female",
   "locale": "de_DE"
}
\end{lstlisting}

In order to get access not only to public, but also to private
information, an access token has to be included in the Graph API
request. To get this token, the accessing entity has to be
authorized and authenticated to do so. A way to achieve this is
explained in section \ref{sec:auth-auth}. When trying to access
non-public information without the necessary access token, an error
is returned. Listing \ref{list:kbeckerfriends} gives an example.


\begin{lstlisting}[caption=JSON object from
  \url{https://graph.facebook.com/681178092/friends},
  label=list:kbeckerfriends]{}
{
   "error": {
      "type": "OAuthException",
      "message": "An access token is required to request this resource."
   }
}
\end{lstlisting}


Besides the Graph API, Facebook's social graph can also be accessed
by using an older API that facebook refers to as its `REST-API'. This API is in the processes of deprecating
\cite{facebookDev3}. Both, the Graph API and the REST-API support
O-Auth 2.0 for authentication and authorization. O-Auth 2.0 is
included in Facebook's \ac{sdks} (see section
\ref{sec:integr-appl-with} and \ref{sec:auth-auth}).




\subsection{Integration of applications with Facebook}
\label{sec:integr-appl-with}
Applications on Facebook are hosted on external severs and are loaded
into a \textit{Canvas Page} by using iFrames. After registration
(\url{http://www.facebook.com/developers/createapp.php}), applications
can be found under the URL \url{http://apps.facebook.com/your_app }.
When the application has existed for some months and 
some users have registered themselves, it can be added to Facebook's application
index. 

Facebook provides several features to support the development
process. \ac{sdks} exist for popular programming languages and provide
sets of pre-defined functions for the purpose of communication with Facebook. For
instance, they include the O-Auth 2.0 dialog which is described in
section \ref{sec:auth-auth}. In
addition, Facebook grants the observation of its own development process by
making its raodmap public and by publishing its buglist. Furthermore, a
testserver exists in order to help developers analyze whether their application
is going to run with planned features from Facebook. 
Facebook provides an \textit{Insights}-feature which makes
statistics about an application accessible. \cite{oreilly}.



\subsection{Authentication and authorization}
\label{sec:auth-auth}
In order to authorize and authenticate, Facebook recommends usage of
O-Auth 2.0 protocol by using the OAuth Dialog included in its
\ac{sdks}. As the application analyzed in this paper and described in section 
\ref{sec:method} implements the O-Auth Dialog from the PHP-\ac{sdk},
this variant will be described in detail. It ought be mentioned,
that Facebook also allows for manual execution of the steps to receive
access tokens.\cite{facebookDevAuth}  Indeed, this is
out of scope for the case of this paper.

The OAuth Dialog performs three steps, i.e. user authentication, 
and authorization as well as authentication of the
application. Whenever these three steps are completed, the application is
issued an \textit{user access token}. If more than basic information
is needed from Facebook, scope parameters can be set.\cite{facebookDevAuth}

Figure \ref{fig:authorization} details how authorization and
authentication take place with OAuthDialog. The \textit{app server}
depicts the server on which the application is actually
running. \textit{Facebook} represents all the Facebook servers, on
which authentication and authorization are performed. The \textit{OAuth
  Dialog} runs on the browser of the client who is going to use the
application.


When clients access the canvas page of the application, they are
redirected to the OAuth Dialog (running on a Facebook server) and
provided with the application's ID. Afterwards, user authentication and
authorization of the application are performed by involving the client and Facebook
only. User authentication is done by logging in on Facebook (if necessary) and
authorization of the application is done by asking the user whether he/she
wants to grant the application access to some data specified by the
application with the scope parameter. One should note, that no
login data are sent to the application. Thus, this is how Facebook
establishes a trust relation. 
If the user authorizes the application, an \textit{authorization code}
will be returned and
the user is redirected to the application to which the
\textit{authorization code} is forwarded. During the registration of the
application, an \textit{application secret} code is
created. In order to authenticate the application, the
\textit{authorization code} and the \textit{application secret} have
to be sent to Facebook's \textit{Graph API Token Endpoint}. Facebook
then returns an \textit{access token}, which can be included in the
URL by which the Graph API can be accessed. However, Facebook's
\ac{sdks} also provide functions to perform calls to the Graph API,
and in this case, the access token is included automatically.

\subsection{Graphs with DOT language}
\label{sec:graph-repr-with}
The \ac{dot} language allows for plain text description of
graphs. There are few programs that can process \ac{dot} graphs. Many
of them are part of the
\textit{Graphviz}\footnote{http://www.graphviz.org}-package. 
The language allows for creation of directed and undirected graphs. Attributes such
colors and shapes for nodes and edges can
be set. The graph shown in figure
\ref{fig:dot} has the \ac{dot}-code given in listing \ref{list:dot}.

\begin{lstlisting}[caption=A very simple DOT-Graph,
  label=list:dot]{}{
 digraph myGraph {
     a -> b;
}
\end{lstlisting}

% Figure DOT 1 and 2
Labels and colors can be added by defining the node in the head of the
graph description. An example is detailed in listing \ref{list:dot2}, and
the corresponding graph is given in figure \ref{fig:dot2}.


\begin{lstlisting}[caption=A very simple DOT-Graph,
  label=list:dot2]{}{
 digraph myGraph {
a [label="Foo", color="blue", shape=box, fontcolor="red"];
     a -> b;
}
\end{lstlisting}

\subsection{Software Metrics}
\label{sec:metrics}
Metrics define a mapping from a `particular measurable entity to a
numerical value' \cite{Lanz06a}. Thus, they allow for analysis and summarization
of properties which oftentimes are not comparable without the metric. In addition, they
facilitate the detection of `outliers in large amounts of data'
\cite{Lanz06a}. Metrics can either be plotted as diagrams or charts, or they
can be used for visualizations. Such are graphs connecting nodes
characterized by metrics. 

Since humans are trained in understanding signs and
pictures, visualizations support the understanding and identification of
`hidden aspects'.\cite{Lanz06a} In this paper, they are
employed to characterize users of the quiz. A tool evaluating the quiz
from the quiz exists, which can put out metrics
as \ac{csv}. In addition, it uses metrics are used to create the graphs in \ac{dot} language.

The following metrics can be printed to \ac{csv} files:

\begin{itemize}
\item A user's number of friends.
\item A user's score.
\item The number of friends who also participated in the quiz.
\item The number of friends who became participants due to
participation of the person analyzed. These persons are regarded as
persons who spread the application.
\end{itemize}

Further metrics are provided by Facebook Insights (see section
\ref{sec:integr-appl-with}). These include the number of Daily
App Installations, the Daily Application Usage, Monthly Active Users,
and the number of users who belong to a specific age group, or to a
specific gender.

\section{Methodology}
\label{sec:method}
In this section, it is detailed how the spreading of an application on
Facebook can be analyzed. At first, the quiz developed and employed for this issue is
explained. Afterwards, a method is described how the set of
participants can be transferred into a graph and how this
graph can be used to identify users who can have contributed to the applications
spreading. After this, the implementation of the quiz is detailed. In
the end of the chapter, it is described how the data were
evaluated. As a tool was developed for this purpose, a focus is put on
the way the tool  operates. The section finished with an explanation of
the way in which the spreading of the application was initialized.

\subsection{The application to be analyzed}
\label{sec:aboutQuiz}
The application to be analyzed is a quiz about dental health fact. It is to be
found at the URL \url{http://apps.facebook.com/zahniquiz}.
Each participant is asked five questions selected
from a set of twenty questions randomly. A question has four answers, of
which either one, two, three or each answer can be
correct. Consequently,
the kind of answers refers to the term \textit{multiple selection}.
A question is scored as a right answer, if the fitting subset was
selected by the user. 
Each question answered correctly scores one. Hence, the
highest score is five. Figure \ref{fig:quiz} gives a screenshot of the quiz. 

The quiz is competitive. After the quiz is completed the score is
displayed, and in addition, they acknowledge how good they were in 
comparison to other participants as well as to their 
friends. As shown in figure \ref{fig:solution} for each question the
correct solution is shown. 

To invite friends to the quiz, profile
pictures of a user's friends are 
printed beyond the quiz. By clicking on such an image,  requests to
participate in the applications 
can be sent to the appropriate friend.


\subsection{User-Graph and identification of people who can have contributed to the application's spreading}
\label{sec:user-graph-ident}
To identify characteristics of people who contributed to the spreading
of the application a graph can be created. In this section, this graph
will be termed User-Graph.

The nodes of the User-Graph represent either people who actually participated in the
quiz or friends of these people. Each node has some attributes: The
user's name, a randomly selected name to make the person anonymous,
and an id. If a person participated  
in the quiz, additional attributes are set. They refer to a
user's birthday, the gender, and information about the results the
user gained in the quiz. Edges are represented by friendships. 

The graph can be used to identify people who can have contributed to
the application's spreading. A method to achieve this is shown in
figure \ref{fig:newNW}: User \textit{a} starts to use
the application and is friend of user \textit{b}, \textit{c}, and
\textit{d}. As user \textit{f}, and \textit{e} are only friends with
user \textit{b}, this user contributed to the application's spreading
very likely. It has to be mentioned that further ways exist to attract user \textit{e} or \textit{f}
to join the application. In this paper it is assumed that virality
is very important for the process of social transmission of an
application. Accepting this assumption, the existence of further ways
to attract people to participate becomes a weak argument.

Identification of people such as user \textit{b} can be achieved as
follows: At first,
the social network of the person who started to spread the application is
considered. For each participant, it is checked if participating
friends exist, who are not a friend of the person who started the
application's spreading. Those people are marked. Then, for each
marked person their friends are analyzed almost in the same way: For
each friend it is checked, if participating friends exist, not
included in the list considered list of friends and not included in
the list of friends of the person who started spreading the
applications. If people are found, they are marked, too. This
procedure is repeated as long as participating friends exist. In order
to avoid an infinite searching for people, for each marked person it
must be stored to which persons they probably spread the
application. If a person would be added twice, no further search is
needed within this part of the application. 

It can be argued, that this method of identification does not consider
the shortest path on the graph. This is correct, but as it is not
known how the spreading took place it does not seem rational to
consider one path only. However, if more than one person is marked due to the
same participants, weights computed with respect to the distance
to the person who initialized the spreading could be used.



\subsection{Implementation}
\label{sec:implementation}
The implementation of the quiz was done using PHP, Java Script, and MySQL.
Facebook's PHP-\ac{sdk} and
JavaScript-\ac{sdk}\footnote{http://developers.facebook.com/docs/}
were employed as well. The application requires
permissions for \textit{user$\_$birthday} and
\textit{publish$\_$stream} (see \ref{sec:auth-auth}). These
permissions permit the application to retrieve a user's birthday and to
send posts to a user's news stream.

The questions and answers of the quiz are stored in a MySQL-database. PHP scripts
handle the creation of the quiz and
the Pear DB module \footnote{http://pear.php.net/} is used
for database queries.

The structure of the Facebook application is related to a
Model-view-controller architecture. \textit{Pear::DB} represents the
model, files stored in folder \textit{templates} act as view
(\textit{quizQuestion.php}, \textit{quiz.php},
\textit{quizResult.php}), and \textit{index.php} assumes the role of
the controller. \textit{library.php} contains a collection of
functions used by the various PHP scripts. \textit{facebook.php} and
scripts from the \textit{facebook-sdk} folder are downloaded from
\url{https://github.com/facebook/php-sdk/} provide the functionality
of Facebook's PHP \ac{sdk}.

The view operates as follows: \textit{quiz.php} checks if the user
completed the quiz, and subject to this, either the quiz is shown or
the pages containing the solution are displayed and the user is
provided with
the following options:  The result can be printed into the news
stream and invitations can be send to friends' Facebook
profiles. For both options Facebook's JavaScript \ac{sdk} is
used. \textit{quizResult.php} contains the code to  
print the correct solution after the quiz is completed and is included
by \textit{quizQuestion.php}. 


\subsubsection{Database schema}
\label{sec:database-schema}
The database queries are performed with the 
Pear\footnote{http://pear.php.net/} library. The following tables exist:
\begin{itemize}
\item \textbf{User} - entries are created for each participating user as
well as for each of a user's friends. A flag details if a user
participated or not.
\item \textbf{Friends} - creates entries for each friendship found in
users' list of friends.
\item \textbf{Questions} - contains the questions of the quiz.
\item \textbf{Answers} - contains the possible answers and a flag to mark them
either right or wrong.
\item \textbf{GivenAnswers} - stores each answer given by a user.
\item \textbf{UserScore} - stores the score a user had in the quiz.
\item \textbf{CorrectAnsweredQuestions} - stores which questions were
answered right by which user, and which were not.
\end{itemize}

\subsubsection{Application flow}
\label{sec:application-flow}
When a user accesses the canvas page of the quiz, first
the \textit{OAuth Dialog} is executed. 
If authorization and authentication are completed
successfully, the user's basic data and the birthday are stored together with
a timestamp in the \textit{User} table. Column \textit{has-used-app}
is set to 1 if a user participated. In
addition each user's list of table is also stored in \textup{User} and
\textit{Friends} table and for these users column
\textit{has-used-app} is set to 0. 

% the quiz is created by randomly choosing
% five questions from the questions table and requiring the answers. To
% make the questions more attractive, the name of an image convenient to
% the questions topic is also stored in the questions table, and this
% image is placed on the left side of each question. Since multiple
% selection is possible, before the answers, which are also queried from
% the MySQL database, checkboxes are placed before each answer. 
Afterwards the quiz is created as mentioned in section
\ref{sec:implementation}.
After completion, the quiz is evaluated and the final score is computed. Because
each answer ought to be stored and multiple answers are
possible, answers of users are stored in  a multidimensional array. 

The given answers are stored in table \textit{GivenAnswers}. In
table \textit{correctAnsweredQuestions} it is stored which questions
were answered right or wrong. This happens to spare complicated
evaluations based on the \textit{GivenAnswers} table later on.
Finally, an average score is computed and stored in the user table.

% In order to make the whole thing challenging, it
% is then computed how high the user's score is compared to the friends
% and to all the other users. 
% In addition, the correct solutions are given, and if a user knew the
% right answer, a green check is placed left to the question. Otherwise,
% a red cross is displayed at the same place. At least, the table with
% the friendships and table \textit{correctAnsweredQuestions} are used
% to identify friends  which answered the question correctly. Images of
% them are placed beyond each question's solution together with the
% information that those friends knew the right solution. 
% In the end, users get the chance to invite friends to the app to
% challenge them. For this reason, each friend's picture is added and
% user can click at them. On click, a pre-defined function from
% Facebook's Java Script \ac{sdk} is called to perform the invitation. 



\subsection{Data evaluation}
\label{sec:data-evaluation}
To evaluate the data retrieved from Facebook and collected when users
participated in the quiz  a tool was developed. It is coded in PHP and
employs MySQL as well as the Pear library to access the
data which were gathered when people participated in the quiz. The
tool allows for output of metrics as \ac{csv} and graphs in \ac{dot}
language\footnote{http://www.graphviz.org/Documentation.php}. These
metrics are used to identify characteristics of people stored in
the \textit{User} table. The graphs are used to visualize the whole
process of the application's spreading. 
Besides usage of the tool, data can also be retrieved using database
queries. Such queries were performed to get the number of people in the user table and
the number of people who actually participated.

\subsubsection{Implementation of the tool for evaluation}
\label{sec:impl-tool-eval}
As mentioned in previous sections, the tool maintains an extract of
Facebook's social graph internally. This graph contains users' basic
information, friendships, and so forth. The internal graph is
extended by data gained through quiz evaluations. This data includes
information about application usage and the score the users
received. The nodes and edges of the internal graph 
are implemented as instances of class \textit{user}. Edges on the
graph represent friendships. These are represented by lists of
references to the appropriate user objects and stored as a user object
attribute.
The whole set of user objects is
maintained by \textit{main.php}, where they are stored as a list.

Before the graph is created, user names can be made anonymous and the
results can be stored in the \textit{user} table to provide
reusability. User names are made anonymous by random selection of
a name from a list of popular baby names from 2009. The list of names was
taken from \url{http://www.ssa.gov/cgi-bin/popularnames.cgi}.   
For each gender hundred names are stored in an array and selected
randomly depending on the user's gender. 
For users whose gender is not known, name and sex are set randomly. 
The name of the person who started spreading the application
originally is set to \textit{Alexandra} by default.

The tool identifies people who can have contributed to the
application's spreading as follows: People who are supposed to have
spread the quiz are found by recursively checking if an user of 
the application has participating friends whose friends also participated
in the quiz, but are neither contained in the analyzed user's list of
friends nor in the list of friends of the person
who spread the application in the beginning. Then, references to the
people to which the user is supposed to have spread the application
are stored in a list. If a reference is added twice, the recursion is stopped.

\subsubsection{Graphs and metrics the tool can put out}
\label{sec:graphs-metrics-tool}
In order to let the tool print the graphs in \ac{dot} language, a parameter
has to be given to the tool via HTTP-Get method. The parameters are:
\verb|spreading|, \verb|tree|, \verb|friendsFilled| and
\verb|friendsScored|. If type \verb|spreading| is selected only users
who participated in the quiz are shown. Users who are supposed to have
spread the application get a red colored node. Beside each user its number
of friends and the average score are mentioned. If
type \verb|tree| is 
selected all the users are shown and nodes of users who participated
in the quiz get red colored nodes. If type \verb|friendsFilled| is selected,
only users who participated in the quiz are considered and the
background color is set corresponding to the number of friends. A user
with the highest number of friends has background color black and the one
person with the lowest number of friends has background color
white. If type \verb|friendsScored| is selected the graph is created
analogous to type \verb|friendsFilled| but the background color is
computed with respect to the score the user had in the quiz averagely. 

To let the tool output \ac{csv}, the \verb|csv| parameter has to be
set to \verb|appUser| or to \verb|noAppUser|. 
If \verb|appUser| is selected, for each user the following metrics are
calculated: The number of friends, the number of friends who
participated in the application, and to how many people a person
probably spread the application.  If \verb|noAppUser| is selected, it
is only computed how many friends these users had who participated in
the quiz. It is returned, how many user had one friend, two friends,
and so on.


% Output options are a \ac{csv}-file containing metrics calculated from
% the internal social graph as well as various graphs in dot language, created
% from the internal social graph, too.
% Properties of
% user objects are in the style of the \textit{user}-table's properties, and a user's
% . The whole set of users is maintained by
% \textit{main.php}, where they are stored in an array. Class \textit{Graph}
% allows for output of various graphs in dot-language created from the
% internal social graph. 

\subsection{Start of the spreading process}
\label{sec:start-spre-proc}
The application is online since December 2010. On 3rd of February
and 15th of March it was  published in the author's news stream. The
message was posted in German and can be translated with `please check
out my new app!'. On 2nd of March, the application was added to
Facebook's application directory. 17 messages were sent to friends of
the author asking them to participate. The messages were sent on
5th and 15th of February, 3rd of March, and 1st of April. Data
about the application's spreading were collected since the application
was registered on Facebook.


\section{Results}
\label{sec:results}
A total of n=52 users participated in the quiz and 6311 users were
stored in the user table. In the two months after publishing, $65.38
\%$ of the participants were friends of the author, whereas $34.63 \%$
were not. $32.7 \%$ of the participants were invited through a message on 
Facebook. $67.31 \%$ of the people came without a direct invitation,
e.g. by seeing somebody use the app on their news feed.


In the following passages of this section it is described which
characteristics users had and when the application was used installed
on used. This information is provided based on Facebook's Insights as well as
based on metrics computed with the tool for data
evaluation. In the end of the section,
the graphs showing the way the application spread are shown.

\subsection{Characteristics Of the Participants}
\label{sec:characteristics}
In this section characteristics about the users who participated in
the quiz are given. In the beginning characteristics retrieved from
Facebook's
\textit{Insights} are described.\cite{facebookInsights} Afterwards
metrics collected from database and by the tool are given. 

\subsubsection{Facebook's Insights}
Facebook Insights `provides Facebook Page owners and Facebook Platform
developers with metrics around their content'.\cite{facebookInsights}
These metrics can be accessed without charges. They can be found in the \textit{My Apps}
section on the Facebook Developer Platform.
Facebook Insights provides metrics for specific time intervals. Users
at specific days are distinguished from Lifetime users, who are users
who did not uninstall the application. As two users are registered as
developers, 50 lifetime users were found by Facebook Insights on 14th of April 2011.

\paragraph{Age and gender}
\label{sec:age-gender}
$60 \%$ of the registered users are female, $40 \%$ of the participants
are male. Table \ref{tab:agegender} shows classification of the users
by age and gender. It can be obtained, that most users ( 68 \%) were at
the age of 25-34. Few participants ($8 \%$) were older, and about a
quarter (24 \%) were younger than 25. The chart given in table \ref{fig:age}
visualizes how many participants of groups of specific ages and
genders installed the application. In the first half of period
analyzed, most users were male. In the second half most users were 
female. Few participants were older than 55 or younger than 18.
% table gender and age

\paragraph{Demography}
\label{sec:demographie}
Table \ref{tab:countries} details where the participants are from and table
\ref{tab:languages} presents the languages in which the participants chose
Facebook to occur.

% table about demography
\paragraph{Internal Referrers}
\label{sec:internal-referrers}
Figure \ref{fig:internalRefs} shows where and how many references to the quiz 
appeared on Facebook in the time of 1st of February 2011 to 14th of April
2011. On 14th and 15th of February, 3rd and 6th, 17th and 27th of
March, and 8th and 11th of April referrer to the application were
posted in Facebook's news stream. On 4th and 7th of April the
application was found through Facebook's search by five persons.

\paragraph{Installations and Usage}
\label{sec:user-response}
Figure \ref{fig:installations} shows how many people used and
installed the application in the time of 1st of February to 14th of
April. After the first post in the news stream on 15th of February, no
increase of installations can be found in the chart. Indeed, after the
second post on 3rd of March, sending of invitations to users, and addition of the application to
Facebook's directory index the number of Daily App Installations shows a
peak. It decreased until 7th of March,
and in the following days, only few installations took
place. Nonetheless, the metric Daily Active Users shows that the quiz
was repeated within these days. On 6th of April the number of
Daily App Installations shows a third peak. This peak took part after
the third wave of invitations. 
Figure \ref{fig:iAs} shows the number of Daily App Installations and the
number of Posts in the News Stream. It can be seen that posts in the
news streams were often followed by few further installations of the application.
Summarizing, it can be said that after each stimulus (post in the news
stream, message to users) an increase of Daily App Installations can
be found. Few days after such a stimulus took place usage decreased in each case.

% As mentioned in section \ref{sec:start-spre-proc}, seventeen friends of the author were
% directly invited to participate in the application. Since each of them
% installed the application, seventeen friends of the author installed the
% application for other reasons. The number of installations per day is
% shown in figure \ref{fig:installations}, the data were exported from
% Facebook Insights. A peak of new installations could be
% found after the request to participate was posted on the news stream
% on 3rd of March 2011. In the following days further installations
% happened. Six of the people who were asked directly to participate
% were not asked on 3rd of March, but afterwards. 7 people were asked
% before.


% \paragraph{Requests}
% The quiz allows for invitation of friends to participate. For this
% issue, users can send requests to their friends. In the time of 18th
% of March to 8th of April four requests
% were sent of which one was accepted. In the month before no friend
% requests were accepted or ignored.




\subsubsection{Metrics collected by the tool}
\label{sec:tool-metrics}
As detailed in section \ref{sec:data-evaluation}, the tool allows for
output of metrics as \ac{csv}. Metrics about users who participated in
the quiz are given in table
\ref{tab:spread}. Table \ref{tab:nouser}
describes user who did not participate. Table \ref{tab:scorespread}
and \ref{tab:nospread} sum up metrics from table \ref{tab:spread}.

From the tables it can be derived that thirty four users belonged to the
network of the author who promoted the application in the beginning. Thus, eighteen
application users came from other networks. Five users did not have any
participating friends. On average, people except the author
who had at least one \textit{app friend} had $2.5$ participating
friends on average. People who probably spread the application to 
one person were on average $3$ friends with who used the quiz, too. People
who probably spread the application to two persons averagely had $5.33$
participating friends. As it is shown in table \ref{tab:scorespread},
the average score of people who are supposed to have contributed to
the application's spreading was higher whenever users were supposed to have spread
the application to more than one person.

% Two tables also located in the addendum describe to how many
% other participants ships existed (see table \ref{tab:spread} and
% \ref{tab:nospread}), and to how many people of a
% \textit{new} network the application was (probably) distributed by the
% user. $26$ participants were on average $2.54$ friends with other
% participants and those, who spread the application, seemed to have
% more connections to other people than those who did not.
% The average score of people connected to others was $3.02$. It was
% higher than the average score of people not connected to others ($2.41$).
% People not connected to further participants had fewer friends on average ($106.8$)
% than the connected participants ($138.83$).

Table \ref{tab:nouser} describes friends of participants, who did not
participate in the quiz. As each friend was stored in the database, these people
occur in table \textit{User} and \textit{Friends}. These people are of
interest as they can have received friend requests
or they may have read about the application in the news
stream. 
It can be obtained, that 11 users exist having 4 or five friends who
use the application. Thus, they have more connections to application
users than these users had on average. Indeed, most people who did not participate in the
application were friends with less than three participants.







\subsection{Graphs detailing the spreading of the quiz}
\label{sec:distribution-quiz}
In this section, graphs illustrating how the
application spread are presented. The original user names collected
from Facebook were made anonymous
 (see section \ref{sec:data-evaluation}). Friendships are
represented by directed edges: If user A has user B in the friend list, an
arrow exists from A to B.

Figure \ref{fig:tree} represents the graph that is created when
parameter \verb|tree| is set. In this case, each user stored in \textit{User}
table becomes a node, and friendships stored in table \textit{Friends}
are represented by directed edges. Names of users who participated in the quiz are distinguished by
red font color. The main purpose of this graph type is visualization
of the whole spreading process. 
The graph gives an imagination of how many 
people on Facebook could have read about the application in their 
news stream, or could have been invited by friends to join the
application. In addition, the graph shows how new networks are
reached and how dense the connections were in the network of the root
of the spreading process.


Figure \ref{fig:spreading} shows the graph that can be obtained when
parameter \verb|spreading| is set. In this case, only users who
participated in the quiz are considered. It can be obtained, that the
users Evelyn, Elli, Carlos, Charlotte, David, Sebastian,
Evan, Thomas, and Gabrielle have several connections among themselves,
thus they belong probably to a common social network. Most people of this
network had a relatively high score. 
Additional smaller networks can be detected in the graph as well. One among the users Sarah, two Evans,
and Brody, and another one among Julian, Khole, David, and Carson. In
these networks, the application spread to people of a new network. In
other cases, one person not connected to such a network was attracted
to join the application. The graph also shows that besides virality,
people not connected to the large network of participants joined the
application, too. As five people found the application through
Facebook search, some may refer to them.
At least, it can be obtained that half of the people who spread the application
had less than hundred friends, which is less than the mean on
Facebook, i.e. 130 friends \cite{facebookStats}.

% In the addendum (section \ref{sec:addendum}), two further graphs are
% shown. In figure \ref{fig:friendsScored}, the background color
% represents a user's average score. User with a darker background had a
% higher score with those having a lighter background color. 
% Figure \ref{fig:friendsFilled} visualizes the amount of friends in the
% by giving users having many friends a dark and those having fewer
% friends a lighter background color. 



% \begin{center}
% \begin{table}
% \label{tab:coruser}
% \begin{tabular}{l | c c c}
% a & a & a & a \\

% \hline
% b & b & b & b
% \end{tabular}
% \caption{dummy}
% \end{table}
% \end{center}





\section{Discussion}
\label{sec:discussion}
In the last section, results about different topics were presented:
The age and gender of the users, the progress of installations,
characteristics of people who participated and who did not, and graphs
have been shown visualizing the whole process. The number of
people who participated was low, but this does not astonish as
Facebook is no more as viral as it used to be. `Top application
developers have essentially become \textit{viral growth scientists}'. \cite{facebookVirality}
This is reflected in Zynga's application City Ville collects
$47.9$ million users within days \cite{cityVille}, while other 
and especially smaller firms invest money
amounts up to $150 000$ \textdollar to promote their
applications\cite{facebookVirality}. 
Nonetheless, the number of 
participants achieved with the quiz is large enough to yield interesting effects about the
spreading in a social network nowadays.

Most people who participated were in the age of twenty five to thirty
four. This effect can be traced back to the age distribution of the
author and the surrounding network. Another reason for the infrequent
participation of teenagers may be further characteristics of these
groups: Teeth from teenagers are less frequently affected with caries than older peoples' teeth
\cite{zahngesundheit}, whereas older people use the Internet less
frequently\cite{internetNutzung}. 

In section \ref{sec:internal-referrers} the internal referrers on
Facebook were described. The number of Daily App Installations and
Posts in the News Stream indicate that some posts in the news stream
let to new participation, while others did not. As this effect was
very small, it is hard to draw conclusions, anyhow. It can just be
suggested that people with specific characteristics existed whose
posts in the news stream attracted new participants. In case of other
users, this did not happen.

People who are supposed to have spread the application had highly differing number of
friends (11 friends up to more than 400). Therefore, this metric can not be used to identify people who
spread the application solely. People who had a score less than 3
spread the application only one time. Since people can have asked
other people to participate, this result suggests that these people
did not invite their friends to participate. People who
were supposed to have spread the application had a score higher than the
average score of all participants, which is $2.84$. This effect can
be interpret as follows: On the one hand, many people want to
present only favourable aspects of themselves, and thus, they wanted
their friends to find their profile picture beyond the questions.
As users not only came by invitation, people who gained a high score
are also likely to have a good or even admirable 
general knowledge. If people with admirable general knowledge were also
admired by their friends, those people's participation could make
their friends so curious that they try the application, too.

People who were supposed to have spread the application were
friends with four other participants on average. Considering figure
\ref{fig:installations} which shows that most installations did not
took part after the first post on Facebook, and remembering, that
more than one participant can have spread the application to the same
\textit{new} participant or network, it can be hypothesized that frequent appearance
posts coming from the same application made people curious and
attracted them to join. 

Finally, it can be summarized that not only
specific characteristics of persons who contributed to the
application's spreading attracted new users. Indeed, further
characteristics such as the number of participants in their network,
the willingness to present the own person advantageously,
frequency of posts from the application in the news stream, and particular
interests of the group appeared to be of relevance.


\section{Conclusion and outlook}
\label{sec:conclusion}
This research provided basic techniques for analysis of the ways in which
applications spread on Facebook. The methodology used bases upon the
assumption that virality plays an important role when applications
spread on Facebook. A mechanism to identify people who contributed to
the spreading process within their social network by analyzing an internal
graph was presented, and results gained by its application were discussed.

Results suggest that varied characteristics make users more likely to
become a person who spreads the application. While such persons could
not be distinguished by a particular number of friends, a user's friend's
willingness to participate seemed to interfere with the following particularities:
People with a worse score did not spread the application, while
people who yielded a score clearly higher than the average did so. The
frequency of posts in the news stream about the application and the
number of participating friends seemed to be related to the attraction
of new users, too. Finally, the results suggest that specific users
exist whose posts in the news stream attracted new people to join
the application, while post from other users do not have this effect. 

Future research should  analyze the described and assumed
mechanisms more intensively and define new metrics to analyze the
defined metrics combinatorially. Furthermore, it can be analyzed if the spreading of
applications also allows for conclusions regarding users' positions in
their social network. For instance, users whose post in the
news stream sufficed solely to gain more participants could point to
opinion leaders, while popular people may retrieve more invitations
than other users. Than, the users' role in their social group could be
used to gain further insight into the spreading of applications on Facebook.




\section*{Acronyms}
\begin{acronym}
%\acronym{JDT}{\emph{Java Development Tools}}
%\acronym{NFC}{\emph{Number of Function Calls}}
\setlength{\itemsep}{-\parsep} 
\acro{SP}[SP]{Standardized Patient}
\acro{SPs}[SPs]{Standardized Patients}
\acro{etext}[e-text]{electronic text}
\acro{pdf}[PDF]{Portable Document File}
\acro{CSVs}[CSVs]{Comma Separated Values }
\acro{ttcn3}[TTCN-3]{Test and Test Control Notation Version 3}
\acro{ttcn}[TTCN]{Tree and Tabular Combined Notation}
\acro{etsi}[ETSI]{European Telecommunications Standards Institute}
\acro{itut}[ITU-T]{International Telecommunication Union}
\acro{3gpp}[3GPP]{Third Generation Partnership Project}
\acro{ims}[IMS]{Ip Multimedia Systems}
\acro{itut}[ITU-T]{International Telecommunication Union}
\acro{iso}[ISO]{International Organization for Standardization}
\acro{sut}[SUT]{System Under Test}
\acro{api}[API]{Application Programming Interfaces}
\acro{3g}[3G]{Third Generation}
\acro{ast}[AST]{Abstract Syntax Tree}
\acro{asts}[ASTs]{Abstract Syntax Trees}
\acro{tci}[TCI]{Test Control Interface}
\acro{tri}[TRI]{Test Runtime Interface}
\acro{mtc}[MTC]{Main Test Component}
\acro{ptc}[PTC]{Parallel Test Component}
\acro{antlr}{Another Tool for Language Recognition}
\acro{trex}[TRex]{TTCN-3 Refactoring and Metrics Tool}
\acro{dot}[DOT]{Drawing of Directed Graphs}
\acro{uml}[UML]{Unified Modeling Language}
\acro{csv}[CSV]{Comma-Separated Values}
\acro{tle}[TLE]{Top Level Element}
\acro{tlecand}[TLEcand]{Top Level Element Candidate}
\end{acronym}
 % load acronyms
%% \section{}
%% \label{}

%% References
%%
%% Following citation commands can be used in the body text:
%% Usage of \cite is as follows:
%%   \cite{key}          ==>>  [#]
%%   \cite[chap. 2]{key} ==>>  [#, chap. 2]
%%   \citet{key}         ==>>  Author [#]

%% References with bibTeX database:

\bibliographystyle{elsarticle-harv}
\bibliography{literature.bib}


%% Authors are advised to submit their bibtex database files. They are
%% requested to list a bibtex style file in the manuscript if they do
%% not want to use model1-num-names.bst.

%% References without bibTeX database:

% \begin{thebibliography}{00}

%% \bibitem must have the following form:
%%   \bibitem{key}...
%%

% \bibitem{}

% \end{thebibliography}

\newpage
\appendix

\begin{figure}
  \includegraphics[height=3in]{../img/fbquiz.png}
 \caption{The quiz.}
\label{fig:quiz}
\end{figure}

\begin{figure}
\label{fig:authorization}
  \includegraphics[width=4in]{../img/fbinteraction3.png}
  \caption{Facebook establishes trust relation by not sending login
    data to app server, but using access tokens instead for
    data exchange with app server.}
\end{figure}


\begin{figure}
  \includegraphics[width=3.5in]{../img/fbsolution.png}
  \caption{Solution of a question.}
\label{fig:solution}
\end{figure}

\begin{figure}
  \includegraphics[height=1.5in]{../img/newNW.png}
 \caption{This figure emphasizes user b who connects the existing
   network of users to a new social networks with
   red border.}
\label{fig:newNW}
\end{figure}


\begin{figure}
  \includegraphics[width=0.4in]{../img/ab.png}
\caption{A very simple graph in \ac{dot}-language.}
\label{fig:dot}
\end{figure}

\begin{figure}
  \includegraphics[width=0.4in]{../img/graphDot.png}
\caption{A very simple graph with node a blue boxed and fontcolor set
  to red.}
\label{fig:dot2}
\end{figure}



\begin{figure}
  \includegraphics[height=2in]{../img/age.png}
 \caption{This figure shows the total number of participants with
   respect to their age and gender in the time of 1st February 2011 to
   14th of April 2011.}
\label{fig:age}
\end{figure}

\begin{figure}
  \includegraphics[height=3in]{../img/internalRefs.png}
 \caption{This figure shows where internal references to the quiz were
   places on Facebook in the time of 1st of February 2011 to
   14th of April 2011.}
\label{fig:internalRefs}
\end{figure}

\begin{figure}
  \includegraphics[height=3in]{../img/installAndstream.png}
 \caption{Figure shows number of Daily App Installations and posts on
   the news stream per day in the time of 1st of February to 14th of
   April 2011.}
\label{fig:iAs}
\end{figure}




\begin{figure}
  \includegraphics[width=5.2in]{../img/tree2.png}
\caption{Figure shows all users stored in the database. Users who
  participated have a red text color.}
\label{fig:tree}
\end{figure}

\begin{figure}
  \includegraphics[width=5in]{../img/spreading5.png}
\caption{Figure shows all users stored in the database. Users who
  participated have a red text color.}
\label{fig:spreading}
\end{figure}

\begin{figure}
  \includegraphics[width=5in]{../img/installations.png}
\caption{Figure shows all users stored in the database. Users who
  participated have a red text color.}
\label{fig:installations}
\end{figure}



% \begin{figure}
%   \includegraphics[width=3.5in]{../img/fbinvite.png}
%   \caption{This screenshot shows how friends can be invited.}
% \label{fig:invite}
% \end{figure}

% \begin{figure}
%   \includegraphics[width=5in]{../img/friendsScored.png}
% \caption{Figure shows all users stored in the database. Users who
%   participated have a red text color.}
% \label{fig:friendsScored}
% \end{figure}

% \begin{figure}
%   \includegraphics[width=5in]{../img/friendsFilled.png}
% \caption{Figure shows all users stored in the database. Users who
%   participated have a red text color.}
% \label{fig:friendsFilled}
% \end{figure}

\newpage

\begin{center}
\begin{table}
\begin{tabular}{l | c c c c c}
Gender\textbackslash Age  & 13-17 & 18-24 & 25-34 & 45-54 & 55+\\
\hline
female  & 2 \% & 18 \% & 36 \% & 4 \% & 0 \%\\
male & 0 \% & 4 \% & 32 \% & 2 \%  & 2 \%\\
\end{tabular}
\caption{Table shows classification of users by age and gender.}
\label{tab:agegender}
\end{table}
\end{center}


\begin{center}
\begin{table}
\begin{tabular}{l | c }
Country  & Visitors \\
\hline
Germany  & 44 \\
United Kingdom & 2 \\
Israel & 1 \\
United States of America & 1 \\
Austria & 1 \\
\end{tabular}
\caption{Table shows where the users are from.}
\label{tab:countries}
\end{table}
\end{center}

\begin{center}
\begin{table}
\begin{tabular}{l | c }
Languages   & Visitors \\
\hline
German  & 34 \\
English (US) & 10 \\
English (UK) & 4 \\
Greek  & 1 \\
Italian & 1 \\
\end{tabular}
\caption{Languages of the participants.}
\label{tab:languages}
\end{table}
\end{center}

\newpage

 \begin{table}
 \begin{tabular}{ c c c c c}
Anon. Name&Friends&App-friends&Spread App to&Average Score	\\
Alexandra & 133& 34& 34 & 4.4625 \\
Evan & 97& 1& 0 & 3.33333 \\
Justin & 90& 2& 0 & 1.28571 \\
Tristan & 332& 2& 0 & 3 \\
Thomas & 396& 3& 0 & 4 \\
Jose & 233& 1& 0 & 4 \\
David & 322& 6& 0 & 1.5 \\
Gianna & 63& 3& 1 & 4 \\
Makayla & 79& 1& 0 & 3 \\
Connor & 102& 3& 0 & 0 \\
Ethan & 91& 4& 1 & 1 \\
Isabelle & 53& 1& 0 & 1 \\
Gavin & 89& 1& 0 & 2 \\
Caleb & 128& 3& 0 & 4 \\
Sarah & 151& 5& 3 & 3 \\
Emma & 72& 1& 0 & 4 \\
Gabrielle & 135& 4& 2 & 3.63636 \\
David & 254& 3& 2 & 4 \\
Emily & 220& 5& 0 & 3 \\
Victoria & 63& 2& 0 & 2 \\
Brian & 126& 2& 0 & 4 \\
Jose & 127& 3& 0 & 3 \\
Thomas & 186& 4& 0 & 4 \\
Evelyn & 42& 5& 0 & 3 \\
Paige & 76& 2& 0 & 1 \\
Hannah & 264& 2& 0 & 1 \\
Charlotte & 445& 5& 0 & 3 \\
Elizabeth & 35& 1& 0 & 3 \\
Carson & 153& 3& 1 & 3 \\
Carson & 20& 2& 1 & 3

\end{tabular}

\caption{Table (part 1/2) details metrics about users who participated in the
  application. Friends detail the number of friends a user had,
  App-friends is the number of friends who also participated in the
  quiz, Spread App To details to how many people a user probably
  spread the application, and Average Score details to the score the
  user averagely had in the quiz. The names were made anonymous.}
 \label{tab:spread}
\end{table}

\newpage

 \begin{table}
 \begin{tabular}{ c c c c c}
Anon. Name&Friends&App-friends&Spread App to&Average Score	\\
Nathan & 121& 1& 0 & 4 \\
Eli & 11& 9& 2 & 4.23077 \\
Isaac & 290& 1& 0 & 1 \\
Alexandra & 130& 1& 0 & 3 \\
Gabriel & 67& 1& 0 & 2 \\
Paige & 16& 1& 0 & 4 \\
David & 50& 1& 0 & 4 \\
Julian & 207& 2& 0 & 2 \\
Carlos & 3 & 3 & 0 & 4.75 \\
Evan & 74& 3& 0 & 4.5 \\
Sebastian & 47& 3& 0 & 4.41667 \\
Xavier & 80& 0& 0 & 4 \\
Evan & 157& 2& 0 & 3 \\
Brody & 136& 2& 0 & 3 \\
Luis & 60& 1& 0 & 0\\
Allison & 13& 0& 0 & 2.5 \\
Tyler & 229& 0& 0 & 0\\
Emily & 53& 0& 0 & 0\\
Khloe & 78& 1& 0 & 4 \\
Katelyn & 167& 1& 0 & 2 \\
Kaitlyn & 10& 1& 0 & 3 \\
Eva & 272& 0& 0 & 4
\end{tabular}
\caption{Table (part 2/2) details metrics about users who participated in the
  application. Friends detail the number of friends a user had,
  App-friends is the number of friends who also participated in the
  quiz, Spread App To details to how many people a user probably
  spread the application, and Average Score details to the score the
  user averagely had in the quiz. The names were made anonymous.}
 \label{tab:spread2}
\end{table}

\newpage

 \begin{center}
 \begin{table}
 \begin{tabular}{ c c}
Number of app friends & Occurrences \\
\hline
0 app friends & 5 \\
1 app friend & 17 \\
2 app friends & 10 \\
3 app friends & 10 \\
4 app friends & 3 \\
5 app friends & 3\\
6 app friends & 2\\
9 app friends & 1\\
34 app friends & 1 \\
\end{tabular}
\caption{Table sums up metrics describing the participants. It
  details how many users had 0 friends who also participated in the
  application, one friend, and so forth.}
 \label{tab:nospread}
\end{table}
\end{center}

 \newpage

\begin{table}
\begin{tabular}{c c c}
Spread to & Concerned People & Average score \\ 
1&	4 & 2.75 \\
2&	3&3.96 \\
3&	1&3 \\
\end{tabular}
\caption{Table details the average score of people who spread the
  application to one, two, or three users, as well as the number of
  concerned people.}
\label{tab:scorespread}
\end{table}

\begin{table}
\begin{tabular}{c c}
Number of friends & Frequency \\
1& 5150\\
2& 240\\
3& 55\\
4& 6\\
5& 5\\
\end{tabular}
\caption{Table details how many friends user had, who did not
  participate in the application.}
\label{tab:nouser}
\end{table}



\end{document}

%%
%% End of file `elsarticle-template-1-num.tex'.

%%% Local Variables: 
%%% mode: latex
%%% TeX-master: t
%%% End: 
